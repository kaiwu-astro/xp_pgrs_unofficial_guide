\chapter{写作指南}
\label{chapter.author-ins}

% v0.1 by Kai.Wu

你用十几分钟写下的经验,可能会节约下后来人几小时甚至几天的时间。欢迎为本攻略做出贡献。正确的投稿姿势有:

\begin{itemize}
    \item 最简单粗暴的,可以直接把稿件发给 Kai.Wu19 at student.xjtlu.edu.cn,虽然管理员可能没有那么开心,因为要费些时间建文件整理上传,但也是可以的。如果你心疼可怜的管理员,可以看看下面的方法。
    \item 如果你有使用Git和交Pull Request的经验,可以直接在\href{https://github.com/kaiwu-astro/xp_pgrs_unofficial_guide}{本手册的GitHub页面}提交PR。\href{https://www.zhihu.com/question/21682976/answer/79489643}{如何交PR可参照这个链接}
    \item 如果不需要特殊排版(就是基本是纯文字),可以直接在\href{https://github.com/kaiwu-astro/xp_pgrs_unofficial_guide/discussions}{本手册的GitHub页面的Discussion里}粘贴或上传你的稿件。
    \item 另外也可以在Overleaf上写。请发邮件到 Kai.Wu19 at student.xjtlu.edu.cn 申请编辑权限。写完过后,请发邮件给管理员把稿件推送到GitHub。其实由于在Overleaf上多人一起编辑搞不好会搞乱项目,所以不是很推荐,但也是可以的。
\end{itemize} 

\vspace{5mm}
注意:对于上面任何一种投稿方法,为避免多人协作弄乱顺序,请每位作者在\texttt{author-folder}文件夹下建立自己的子目录,在里面新建\texttt{tex}文件写作,第一行直接从\texttt{section}命令开始,其后写正文。写完了过后,在正确的章节里,用以下代码把你的稿件插入到\texttt{chapter}文件夹下的章节里
\begin{lstlisting}
    \include{你的tex文件路径}
\end{lstlisting} 

\vspace{5mm}
好了,就这么点要说的,开始帮助学弟学妹吧。如果要一些高级的编辑方式,可参照项目GitHub或Overleaf的\texttt{author-guide}目录下的模板手册和例子。

属不署名都可以。可以匿名,也可以把姓名邮箱院系入学年份都写上。你如果愿意的话把生辰八字、出生年月、房产车产详情附上也是完全可以的,这样甚至可以问问PGR Society能不能安排相亲(雾)。

最后必须要提醒:您不得发布任何违反中华人民共和国法律、西交利物浦大学博士生守则、\href{https://www.liverpool.ac.uk/aqsd/academic-codes-of-practice/pgr-code-of-practice/}{利物浦大学博士生守则}和其他任何适用规定的内容。您需要对你发布的内容负责,PGR Society无法对内容的准确性做担保或审核。由您发布的内容导致的任何纠纷,PGR Society社团和其他作者不承担任何连带责任。