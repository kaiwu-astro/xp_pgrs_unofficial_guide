\section{学术研究是怎么发生的}
\label{section.how-research-works}

\subsection{经费}
学术经费来源大致有以下三种:
\begin{itemize}
    \item 政府机构及基金会赞助项目。国内有国务院、各省、各市提供的面向项目和面向个人的资助,国外也有政府支持的学会提供。这类项目一般金额大、影响力高,成功的申请记录对申请同一体系的项目有重要帮助。
    \item 企业合作项目。一般由项目负责人与企业接洽获得,大多数项目更关心落地。项目金额、完成度、影响力大小不一。
    \item 校内研究支持项目(RDF 等)。由校内 Faculty 申请,经学院筛选后由某委员会审核。项目数量不多、金额不小、周期不长,对职业初期比较重要。
\end{itemize}

学术经费并不直接进入项目负责人的私人银行账户,而是进入学校的科研经费管理账户。经费支出需符合指定的支出范围并受学校监管,常见的支出项包括采购、会议相关开支、劳务等。

当一次获得的学术经费超过某一额度后,导师可以获得跟经费数量对应的博士生奖学金(PGRS)名额。

\begin{flushright}
    (2024年10月1日 by Shiyao Zhang)
\end{flushright}


\subsection{期刊及会议}
期刊及会议需要至少一个学术团体负责活动的举办。重要、活跃的研究者在学术团体中承担工作,如编辑、审稿人等,有助于吸引投稿参与。与之对应地,在学术团体中承担工作在一定程度上反映着研究者的声誉。

出版商发行出版物、举办会议需要资金支持。一般来说越重要、有影响力、有行业价值的期刊及会议,越容易找到赞助,需要作者支付的出版费用越少,活动的规模、丰富度以及质量越高。

期刊及会议的影响力在很大程度上受发表文章的被引用次数影响。越时髦的话题、重要的工作、知名的作者、重要的期刊,越容易被人阅读、引用。为了活动的可持续性,举办方也会优先考虑这些因素。

很多研究领域的研究者总数很小,参加会议能经常碰到至少读过论文的人,甚至审稿人、viva reviewer。

\begin{flushright}
    (2024年10月1日 by Shiyao Zhang)
\end{flushright}


\subsection{社会效益}
学术研究的一种意思是,在人类认知的极限处创造新知识。从这个角度看,学术研究对社会终将是有益的。

另外听说人文社科也很喜欢做社会影响力方面的工作,有机会再邀请人来写这部分。

不过考虑到我们还是活在一个商业社会,社会效益很多时候几乎直接等价于成果值多少钱,或者潜力多大。这也是我们物理空间上所处的环境。

\begin{flushright}
    (2024年10月1日 by Shiyao Zhang)
\end{flushright}


\subsection{导师}
不同的导师特点都很不同,但大概可以分为应用倾向和前沿倾向。应用倾向的导师一般会有更多的来自企业或政府的落地项目,或一些偏应用的来自政府的资助项目;前沿倾向的导师一般会有更多的重要工作发表。两者倒不是绝对矛盾的,只是时间、精力有限,总会有一定倾向。相对来说应用的机会数量多些,前沿需要的项目资金体量大些。由于导师的倾向不同,其所发展的各类资源也就不同,我们接触到的工作指导也会不同。

基于上面的几点,你大概能同意,很大概率我们会成为各自导师的样子。

另外对于人文社科,我听说的是几乎需要导师全程培养,有机会再邀请人来写这部分。

\begin{flushright}
    (2024年10月1日 by Shiyao Zhang)
\end{flushright}