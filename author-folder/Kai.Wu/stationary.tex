\section{免费办公用品}
除了大笔的会议经费之外,最让大家省钱的应该就是免费的办公用品了。学校很多楼里都有Stationery room(办公用品领用室),里面所有东西都可以领取。抽纸、笔、马克笔、荧光笔、各种收纳架子、胶带胶水订书机,基本你能想到的都有。

哪些地方有stationery room:
\begin{itemize}
    \item MB一楼,每周二/周四,上午9-11点,下午1-5点
    \item BS一楼(房间号?谁知道房间号来补充下)(开放时间?)
    \item 其他楼?请读者补充
\end{itemize}

如何领取:
\begin{enumerate}
    \item 第一次领取前,先直接跑到stationery room,问工作人员可以领哪些,ta会给你一个册子。拍照留存每一页内容,这样下次就知道可以拿什么了
    \item 下载最新的 (CMO) Office Supplies Application Form,目前最新版本是V3。新版本发布后老版本一般是不能用的。如果这个过期了,更新的版本我们在哪里下载呢?很遗憾我们没法下载,一般方法是,问你导师要,导师一般可以在学校的box网盘里面找到。这里提供一个我在用的\url{https://cowtransfer.com/s/e7089cbffe9c43},也可在本项目的GitHub里(\href{https://github.com/kaiwu-astro/xp_pgrs_unofficial_guide/tree/main/fileshare}{链接})找到。
    \item 如实填写内容。其中第一列[办公用品名称]一定要和上面那个册子里的物品名称一样。每次不能拿太多,拿东西的工作人员会审查。抽纸每次只能拿一盒。笔一般是按支拿,不能按盒拿(但是你可以写拿12支,就是1盒)。右边的理由,如实简单填写,比如抽纸-日用;笔-演算;订书机-整理;胶水-报销。最后,要你导师签字。电子签名应该不能用,也不能导师签完过后复印(这些都是之前版本常用的花招hhh)。
\end{enumerate}


\begin{flushright}
(2022年10月12日 by Kai Wu)
\end{flushright}